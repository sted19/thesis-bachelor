\chapter{Risultati}
In questo capitolo saranno mostrati alcuni risultati ottenuti dal meccanismo di riconoscimento facciale;
sono state effettuate due tipologie di test:
\begin{itemize}
	\item test su un dataset di 40823 immagini e 106 classi (106 diversi soggetti) in modo da testare la bontà del classificatore;
	\item test su poche immagini in modo da testare il reale funzionamento dell'applicazione.
\end{itemize}

\section{Test e scoring}
Il dataset utilizzato è quello di YouTube faces di \cite{wolf2011face}. Il dataset contiene immagini di dimensioni ridotte di volti, il più delle volte ben visibili e in condizioni di luce accetabili. In figura \ref{test_images} si può osservare la tipologia delle immagini che compongono il dataset. 

\begin{figure}
	\centering
	\begin{subfigure}{0.1 \textwidth}
		\includegraphics[width=0.8\linewidth]{./data/images/dataset_no_crop/0.jpg}
	\end{subfigure}
	\begin{subfigure}{0.1 \textwidth}
		\includegraphics[width=0.8\linewidth]{./data/images/dataset_no_crop/1.jpg}
	\end{subfigure}
	\begin{subfigure}{0.1 \textwidth}
		\includegraphics[width=0.8\linewidth]{./data/images/dataset_no_crop/2.jpg}
	\end{subfigure}
	\begin{subfigure}{0.1 \textwidth}
		\includegraphics[width=0.8\linewidth]{./data/images/dataset_no_crop/3.jpg}
	\end{subfigure}
	\begin{subfigure}{0.1 \textwidth}
		\includegraphics[width=0.9\linewidth]{./data/images/dataset_no_crop/4.jpg}
	\end{subfigure}
	\begin{subfigure}{0.1 \textwidth}
		\includegraphics[width=0.8\linewidth]{./data/images/dataset_no_crop/5.jpg}
	\end{subfigure}
	\begin{subfigure}{0.1 \textwidth}
		\includegraphics[width=0.8\linewidth]{./data/images/dataset_no_crop/6.jpg}
	\end{subfigure}
	\begin{subfigure}{0.1 \textwidth}
		\includegraphics[width=0.8\linewidth]{./data/images/dataset_no_crop/7.jpg}
	\end{subfigure}
	\begin{subfigure}{0.1 \textwidth}
		\includegraphics[width=0.8\linewidth]{./data/images/dataset_no_crop/8.jpg}
	\end{subfigure}
	\begin{subfigure}{0.1 \textwidth}
		\includegraphics[width=0.9\linewidth]{./data/images/dataset_no_crop/9.jpg}
	\end{subfigure}
	\begin{subfigure}{0.1 \textwidth}
		\includegraphics[width=0.8\linewidth]{./data/images/dataset_no_crop/10.jpg}
	\end{subfigure}
	\begin{subfigure}{0.1 \textwidth}
		\includegraphics[width=0.8\linewidth]{./data/images/dataset_no_crop/11.jpg}
	\end{subfigure}
	\begin{subfigure}{0.1 \textwidth}
		\includegraphics[width=0.8\linewidth]{./data/images/dataset_no_crop/12.jpg}
	\end{subfigure}
	\begin{subfigure}{0.1 \textwidth}
		\includegraphics[width=0.8\linewidth]{./data/images/dataset_no_crop/13.jpg}
	\end{subfigure}
	\begin{subfigure}{0.1 \textwidth}
		\includegraphics[width=0.9\linewidth]{./data/images/dataset_no_crop/14.jpg}
	\end{subfigure}
	\begin{subfigure}{0.1 \textwidth}
		\includegraphics[width=0.8\linewidth]{./data/images/dataset_no_crop/15.jpg}
	\end{subfigure}
	\begin{subfigure}{0.1 \textwidth}
		\includegraphics[width=0.8\linewidth]{./data/images/dataset_no_crop/16.jpg}
	\end{subfigure}
	\begin{subfigure}{0.1 \textwidth}
		\includegraphics[width=0.8\linewidth]{./data/images/dataset_no_crop/17.jpg}
	\end{subfigure}
	\begin{subfigure}{0.1 \textwidth}
		\includegraphics[width=0.8\linewidth]{./data/images/dataset_no_crop/18.jpg}
	\end{subfigure}
	\begin{subfigure}{0.1 \textwidth}
		\includegraphics[width=0.9\linewidth]{./data/images/dataset_no_crop/19.jpg}
	\end{subfigure}
	\begin{subfigure}{0.1 \textwidth}
		\includegraphics[width=0.8\linewidth]{./data/images/dataset_no_crop/20.jpg}
	\end{subfigure}
	\begin{subfigure}{0.1 \textwidth}
		\includegraphics[width=0.8\linewidth]{./data/images/dataset_no_crop/21.jpg}
	\end{subfigure}
	\begin{subfigure}{0.1 \textwidth}
		\includegraphics[width=0.8\linewidth]{./data/images/dataset_no_crop/22.jpg}
	\end{subfigure}
	\begin{subfigure}{0.1 \textwidth}
		\includegraphics[width=0.8\linewidth]{./data/images/dataset_no_crop/23.jpg}
	\end{subfigure}
	\begin{subfigure}{0.1 \textwidth}
		\includegraphics[width=0.9\linewidth]{./data/images/dataset_no_crop/24.jpg}
	\end{subfigure}
	\begin{subfigure}{0.1 \textwidth}
		\includegraphics[width=0.8\linewidth]{./data/images/dataset_no_crop/25.jpg}
	\end{subfigure}
	\begin{subfigure}{0.1 \textwidth}
		\includegraphics[width=0.8\linewidth]{./data/images/dataset_no_crop/26.jpg}
	\end{subfigure}
	\caption{Immagini di test prese dal dataset YouTube faces \cite{wolf2011face}}
	\label{test_images}
\end{figure}

Prima di utilizzare le immagini per la classificazione è stato applicato un algoritmo di \text{face detection} ed è stato effettuato il crop della zona del volto, successivamente è stata utilizzato un metodo di estrazione di features per trasformare le immagini in array di features. Sono state utilizzate 40823 immagini di 106 soggetti diversi; il dataset è stato suddiviso in un insieme di train costituito di 30617 immagini e un insieme di test di 10206. Le immagini di train sono state fornite al classificatore di tipo kNN di scikit-learn e poi è stata calcolata l'accuratezza ottenuta \cite{scikit-learn}. Di seguito il codice utilizzato per effettuare il test:

\begin{lstlisting}
X_train, X_test, y_train, y_test = 
model_selection.train_test_split(feature_vectors,labels)                       
clf = neighbors.KNeighborsClassifier(5)
clf.fit(X_train, y_train)
accuracy = clf.score(X_test, y_test)
\end{lstlisting}
L'accuratezza ottenuta dal classificatore è di 0.9778561630413483.


\section{Test di funzionamento}

Per testare il funzionamento dell'applicazione sono state utilizzate solo 10 immagini con 5 soggetti in tutto: l'applicazione si troverà sempre ad operare con poche immagini di pochi soggetti, quindi è stato necessario considerare che il dataset possa essere composto anche da così poche immagini; immagini dello stesso volto sono diverse perché il volto non si trova nella stessa posizione per entrambe o perché la persona raffigurata ha modificato la sua espressione facciale. Sono state quindi comparate le distanze fra diverse immagini e immagini dello stesso volto catturato in diversi momenti. In figura \ref{try_images} sono mostrate le immagini utilizzate. 

\begin{figure}
	\centering
	\begin{subfigure}{0.2 \textwidth}
		\includegraphics[width=0.8\linewidth]{./data/images/dataset/0.jpg}
		\caption{1a}
	\end{subfigure}
	\begin{subfigure}{0.2 \textwidth}
		\includegraphics[width=0.8\linewidth]{./data/images/dataset/1.jpg}
		\caption{1b}
	\end{subfigure}
	\begin{subfigure}{0.2 \textwidth}
		\includegraphics[width=0.8\linewidth]{./data/images/dataset/2.jpg}
		\caption{2a}
	\end{subfigure}
	\begin{subfigure}{0.2 \textwidth}
		\includegraphics[width=0.8\linewidth]{./data/images/dataset/3.jpg}
		\caption{2b}
	\end{subfigure}
	\begin{subfigure}{0.2 \textwidth}
		\includegraphics[width=0.8\linewidth]{./data/images/dataset/4.jpg}
		\caption{3a}
	\end{subfigure}
	\begin{subfigure}{0.2 \textwidth}
		\includegraphics[width=0.8\linewidth]{./data/images/dataset/5.jpg}
		\caption{3b}
	\end{subfigure}
	\begin{subfigure}{0.2 \textwidth}
		\includegraphics[width=0.8\linewidth]{./data/images/dataset/6.jpg}
		\caption{4a}
	\end{subfigure}
	\begin{subfigure}{0.2 \textwidth}
		\includegraphics[width=0.8\linewidth]{./data/images/dataset/7.jpg}
		\caption{4b}
	\end{subfigure}
	\begin{subfigure}{0.2 \textwidth}
		\includegraphics[width=0.8\linewidth]{./data/images/dataset/8.jpg}
		\caption{5a}
	\end{subfigure}
	\begin{subfigure}{0.2 \textwidth}
		\includegraphics[width=0.8\linewidth]{./data/images/dataset/9.jpg}
		\caption{5b}
	\end{subfigure}
	\caption{Immagini di test ottenute dal dataset YouTube faces applicando un algoritmo di \textit{face detection} e successivo crop del volto \cite{wolf2011face}}
	\label{try_images}
\end{figure}

Il test è stato effettuato prendendo ogni immagine fra le presenti in figura \ref{try_images} e effettuando il riconoscimento facciale utilizzando come dataset tutte le altre immagini, inclusa quella da riconoscere. Di seguito è riportata la stampa effettuata dal programma con le distanze clacolate:
\begin{lstlisting}
array([[0.        , 0.24959708, 0.90029748, 0.901967  ]] 
array([[0, 1, 6, 7]]))

array([[0.        , 0.24959708, 0.87065946, 0.91801676]]
array([[1, 0, 6, 7]]))

array([[0.        , 0.57770619, 0.88179605, 0.88401503]]
array([[2, 3, 9, 8]]))

array([[0.        , 0.57770619, 0.76117377, 0.76989079]]
array([[3, 2, 8, 9]]))

array([[0.        , 0.56632163, 0.87791262, 1.01528504]]
array([[4, 5, 8, 6]]))

array([[0.        , 0.56632163, 1.05295989, 1.06557215]]
array([[5, 4, 7, 6]]))

array([[0.        , 0.3715512 , 0.87065946, 0.90029748]]
array([[6, 7, 1, 0]]))

array([[0.        , 0.3715512 , 0.901967  , 0.91801676]]
array([[7, 6, 0, 1]]))

array([[0.        , 0.46096002, 0.76117377, 0.87791262]]
array([[8, 9, 3, 4]]))

array([[0.        , 0.46096002, 0.76989079, 0.88179605]]
array([[9, 8, 3, 2]]))

\end{lstlisting}
Ogni coppia di due righe è il risultato conseguito da un'immagine, nell'ordine che hanno in figura \ref{try_images}. Come è facile notare, la distanza di ogni immagine da se stessa è 0, mentre la distanza dall'altra immagine con lo stesso volto è un valore relativamente piccolo, generalmente inferiore a 0.6. Le distanze che le immagini presentano dalle altre che non raffigurano lo stesso soggetto sono tendenzialmente elevate e, in questo caso ristretto di test, non inferiori a 0.75. A partire da questi risultati si può determinare che una soglia accettabile sia 0.6, o anche 0.5 se si vuole ottenere maggiore sicurezza evitando falsi positivi, sacrificando alcuni risultati che, seppur corretti, presentano un valore di distanza superiore alla soglia. 









