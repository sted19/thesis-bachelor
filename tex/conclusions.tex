\chapter{Conclusioni}
L'applicazione sviluppata è funzionale nel suo compito di automatizzare e rendere smart l'accesso a smart-homes, sebbene ci siano ancora miglioramenti da apportare per renderla più accurata. Il meccanismo di rilevamento facciale è rapido e preciso; il meccanismo di riconoscimento facciale invece, è un po' meno accurato e ha ancora molti margini di miglioramento in quanto, allo stato attuale, è ben funzionante in casistiche semplici, ma, quando le immagini diventano più difficili da classificare, potrebbe risultare non sufficientemente sicuro poiché la differenza di distanza calcolata fra immagini che non raffigurano lo stesso volto e immagini che invece lo raffigurano non è così elevata da garantire la scelta di un valore di soglia ottimale senza accettare dei compromessi scomodi: se il valore di soglia determinato è troppo alto si rischia di ottenere falsi positivi, il che è totalmente inaccettabile per un'applicazione di riconoscimento facciale; se invece il valore di soglia è troppo basso il rischio è che l'applicazione riesca troppo raramente a calcolare una distanza così piccola, anche tra immagini molto simili fra loro, vanificando la presenza dello stesso riconoscimento facciale. Ci si prospetta di ottenere futuri miglioramenti lavorando sulle modalità  di estrazione delle feature e introducendo un meccanismo di \textit{face normalization}.